\chapter{Software}

    \section{Projektstruktur}
    Die Projektstruktur orientiert sich an PlatformIO: \\
    \(\rightarrow\) Command: \texttt{tree -L 2} \\
    \vspace{0.5cm}
    \begin{minipage}{0.48\linewidth}
            \centering
            \includegraphics[width=8cm]{projDir}
            \label{fig:projDir}
    \end{minipage}
    \hfill
    \begin{minipage}{0.48\linewidth}
        \raggedright
        \begin{itemize}
            \item custom Libraries werden im \lstinline{lib/} 
            Verzeichnis abgelegt
            \item Libery-unabhängige Header-Dateien 
            werden im \lstinline{include/} Verzeichnis abgelegt
            \item Libery-unabhängige Source-Dateien 
            werden im \lstinline{src/} Verzeichnis abgelegt
        \end{itemize}
    \end{minipage}
    \vspace{0.5cm}

    Während Libraries, die speziell für das Projekt entwickelt oder abgeändert wurden,
    im \lstinline{lib/} Verzeichnis abgelegt werden, werden Libraries,
    die von PlatformIO bereitgestellt werden, automatisch heruntergeladen
    und im verstecktem \lstinline{.pio/} Verzeichnis abgelegt. Diese sind im
    \lstinline{platformio.ini} File spezifiziert.

    \section{FaRiLib}
    Die Libary soll als Grundgerüst für die Implementierung aller Smarthome-Geräte
    dienen. Daher ist der größte Teil der geschriebenen Funktionen Teil der 
    \textit{FaRiLib} Library. Diese Funktionen werden nach Kategorien in verschiedene
    Dateien aufgeteilt (Zum Beispiel: \lstinline{FaRiLib/src/Display.cpp} und 
    \lstinline{FaRiLib/src/ESP-NOW.cpp}). Die selbe Gliederung gilt für die 
    dazugehörigen Header-Dateien (Zum Beispiel: \\ \lstinline{FaRiLib/include/Display.h} und 
    \lstinline{FaRiLib/include/ESP-NOW.h}).

    \section{ESP32-S3 Konfiguration}
    Die Eigenschaften des ESP32-S3 werden in dem \lstinline{platformio.ini}
    File konfiguriert. Als Basis der Konfiguration dient
    das \lstinline{esp32-s3-devkitc-1.json} File, welches die
    Basiskonfiguration für ein ESP32-S3 Devkit bereitstellt
    und standardmäßig von PlatformIO verfügbar ist. \par
    
    Nun müssen folgende Flags im \textit{platformio.ini} File
    überschrieben werden:

    \begin{lstlisting}[style=cppCode]
    board_upload.flash_size = 16MB
    board_build.partitions = default_16MB.csv
    \end{lstlisting}

    Nun sollte der Upload eines Programms auf den 
    ESP-Chip möglich sein.

        \subsection{Serielle Kommunikation}
        Für Entwicklungszwecke ist es sinnvoll, die 
        serielle Kommunikation zu aktivieren. Hierfür
        muss, mithilfe spezieller \texttt{build\_flags}
        im \texttt{platformio.ini}, die serielle Kommunikation über die USB-Schnittstelle
        aktiviert werden:

        \begin{lstlisting}[style=cppCode]
    build_flags =
        -DARDUINO_USB_MODE=0
        -DARDUINO_USB_CDC_ON_BOOT=1
        \end{lstlisting}

            \subsubsection{Problematik}
            Obwohl die serielle Kommunikation aktiviert ist,
            funktioniert die Kommunikation nicht auf jedem
            ESP32-S3. Da PlatformIO's offizielle 
            Dokumentation \cite{PlatformIO_build_flags}
            unzureichende Informationen bereitstellt und
            Debugging auch über das angeschlossene Display
            möglich ist, wurde die serielle Kommunikation
            nicht weiter verfolgt.



    \section{User Interface}
        \subsection{SquarelineStudio}
        SquarelineStudio ist eine Software, die es ermöglicht, ein 
        User Interface mithilfe eines Drag-and-Drop-Editors zu erstellen und
        in C-Code zu exportieren. Dieser Code kann dann dann zusammen mit den 
        beiden Libraries \textit{TFT\_eSPI} und \textit{lvgl} in PlatformIO
        integriert werden.

            \subsubsection{TFT\_eSPI}
            TFT\_eSPI ist eine Library für Grafik und Fonts auf einem TFT-Display.
            Sie ist mit vielen verschiedenen Controllern kompatibel und bietet
            viele Funktionen, um verschiedene TFT-Display anzusteuern.
            Sie ist eine Hälfte des Grundgerüsts für die Darstellung von Grafiken 
            in SquarelineStudio.

            \subsubsection{lvgl}
            lvgl ist eine Library für die Darstellung von flexiblen Grafiken
            auf vielen Platformen, darunter auch dem Arduino Framework.
            Zusammen mit TFT\_eSPI bildet sie das Grundgerüst für die Darstellung
            von Grafiken in SquarelineStudio.
        \subsection{Touch Control}
        Die \textit{TFT\_eSPI} Library bietet auch die Möglichkeit, Touch-Events
        zu registrieren und für das \textit{SquarelineStudio} User Interface zu
        verarbeiten. Diese Funktion ist allerdings nur für über SPI angesteuerte
        Displays verfügbar.
            \subsubsection{XPT2046\_Touchscreen Library}
            Um den Touch seriell einzulesen, wird die \textit{XPT2046\_Touchscreen}
            Library verwendet. 

            \begin{lstlisting}
                #define CS_PIN  10

                XPT2046_Touchscreen ts(CS_PIN);
                TS_Point p;
            \end{lstlisting}
    \section{Kommunikation}
    \subsection{MQTT}
    Um eine Kommunikation zu ermöglichen, ist eine 
    MQTT-Library notwendig. Hierfür wurde "PubSubClient" von 
    "knolleary" verwendet. Hierbei handelt es sich um eine 
    simple Library, die nur das Nötigste implementiert, um
    einen überschaubaren Overhead zu gewährleisten.
        \subsubsection{Nachteile}
        Da MQTT eine WLAN-Verbindung für jeden ESP32
        benötigt und somit die Mobilität des Controlpanels
        deutlich einschränkt ist wurde entschieden, dass 
        ESP-NOW eine bessere Alternative darstellt.
    
    \subsection{ESP-NOW}
    Für eine Implementierung von ESP-NOW wurden die Libraries
    \texttt{esp-now.h} und \texttt{esp\_wifi.h} von
    „Espressif“ verwendet. Diese sind offiziell unterstützt
    und bieten eine einfache Möglichkeit, um eine Kommunikation
    zwischen ESP32's zu ermöglichen.





\section{User Interface}
        \subsection{SquarelineStudio}
        SquarelineStudio ist eine Software, die es ermöglicht, ein 
        User Interface mithilfe eines Drag-and-Drop-Editors zu erstellen und
        in C-Code zu exportieren. Dieser Code kann dann zusammen mit den 
        beiden Libraries \textit{TFT\_eSPI} und \textit{lvgl} in PlatformIO
        integriert werden.

            \subsubsection{TFT\_eSPI}
            TFT\_eSPI ist eine Library für Grafik und Fonts auf einem TFT-Display.
            Sie ist mit vielen verschiedenen Controllern kompatibel und bietet
            viele Funktionen, um verschiedene TFT-Display anzusteuern.
            Sie ist eine Hälfte des Grundgerüsts für die Darstellung von Grafiken 
            in SquarelineStudio.

            \subsubsection{lvgl}
            lvgl ist eine Library für die Darstellung von flexiblen Grafiken
            auf vielen Plattformen, darunter auch dem Arduino Framework.
            Zusammen mit TFT\_eSPI bildet sie das Grundgerüst für die Darstellung
            von Grafiken in SquarelineStudio.
        \subsection{Touch Control} \label{sec:touch}
        Die \textit{TFT\_eSPI} Library bietet auch die Möglichkeit, Touch-Events
        zu registrieren und für das \textit{SquarelineStudio} User Interface zu
        verarbeiten. Diese Funktion ist allerdings nur für über SPI angesteuerte
        Displays verfügbar.
            \subsubsection{XPT2046\_Touchscreen Library}
            Um den Touch seriell einzulesen, wird die \textit{XPT2046\_Touchscreen}
            Library verwendet. 

\begin{lstlisting}[style=cppCode]
    #define CS_PIN  10

    XPT2046_Touchscreen ts(CS_PIN);
    TS_Point p;
\end{lstlisting}

            \subsubsection{Touch mit tft\_eSPI kuppeln} \label{touch_to_tft}
            Um die Touch-Events an tft\_eSPI weiterzuleiten, muss die Library grundlegend \\ 
            umgeschrieben werden.\\
            Zunächst muss das User\_Setup.h folgendermaßen bearbeitet werden:

\begin{lstlisting}[style=cppCode]
    //Einkommentieren
    #define TOUCH_CS 10 
\end{lstlisting}

            
            \begin{minipage}{\linewidth}
            Ebenso muss im "tft\_eSPI.h" folgendes editiert werden:
\begin{lstlisting}[style=cppCode]
    #ifdef TOUCH_CS
    //CHANGE THIS LINE
    #if 0
        #if !defined(DISABLE_ALL_LIBRARY_WARNINGS)
        #error >>>>------>> ...
        #endif
    #else
        #include "Extensions/Touch.h"
    #endif
    #else
        #if !defined(DISABLE_ALL_LIBRARY_WARNINGS)
        #warning >>>>------>> ...
        #endif
    #endif
\end{lstlisting}\end{minipage}
            
            

            \begin{minipage}{\linewidth}
            Nun kann der Touch in der my\_touchpad\_read() Funktion der tft\_eSPI Library
            übergeben werden.
            \begin{lstlisting}[style=cppCode]
    void my_touchpad_read
    (lv_indev_drv_t*indev_driver, lv_indev_data_t*data)
    {
        uint16_t touchX = 0, touchY = 0;
    
        if( !ts.touched() || p.z < 1950 )
        {
            data->state = LV_INDEV_STATE_REL;
        }
        else
        {
            data->state = LV_INDEV_STATE_PR;
    
            p = ts.getPoint();

            /*Set the coordinates*/
            data->point.x = p.x;
            data->point.y = p.y;
    
            Serial.print( "Data x " );
            Serial.println(data->point.x-700);
    
            Serial.print( "Data y " );
            Serial.println(data->point.y);
        }
    }
            \end{lstlisting}\end{minipage}

            \subsubsection{Touch wurde nicht implementiert}
            Die tft\_eSPI Library verhindert duch die in section \ref{touch_to_tft} 
            beschriebenen \lstinline{#if} Makros, die Implementierung des Touches.
            Trotz mehrerer Versuche, die Library zu modifizieren, wirf die Library
            immer wieder \lstinline{include-Fehler} in dem dann inkludiertem Touch.h File. \\~\\
            

            


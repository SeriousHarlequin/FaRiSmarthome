\chapter{Einführung}

Das Control-Panel soll als Erweiterung eines Smarthomes dienen, also
mit einem bereits aufgebauten System verbunden werden und in der Lage sein, 
über die im Smarthome bereits integrierte ESP-NOW Schnittstelle mit 
integrierten Geräten zu kommunizieren und den Status dieser Geräte 
abzufragen und anzupassen.

    \section{Motivation}
    Der Sinn des Control-Panels ist es, die Bedienung des Smarthomes zu vereinfachen.
    Es soll eine Anlaufstelle für die Steuerung des Smarthomes sein, die kein
    herkömmliches Heimnetzwerk benötigt, sondern direkt mit den Geräten kommuniziert.
    Des Weiteren muss so kein Browser oder App geöffnet werden, um das Smarthome zu 
    steuern.

    \section{Das Smarthome}
    Das Smarthome, mit dem das Control-Panel kommuniziert, ist ein bereits 
    bestehendes System, das als Diplomarbeit derselben Autoren entwickelt wurde.
    Es besteht aus mehreren ESP32-Mikrocontrollern, die über ein ESP-NOW-Netzwerk
    miteinander kommunizieren. Die Geräte sind in der Lage, verschiedene 
    Sensor-Daten zu erfassen, sowie Aktoren zu steuern. Als Alternative zum
    Control-Panel kann das Smarthome auch über eine Web-Oberfläche gesteuert werden.


        
        
        

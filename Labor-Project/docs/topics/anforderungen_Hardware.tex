\section{Hardware}


    \subsection{MCU}
    Es wurde sich für einen ESP-32 S3 als Microcontroller entschieden.
    Dieser ist mit diveresers Funktionien(Bluetooth, WLAN, ESPNOW ...) 
    sowie genügend Port-Pins ausgestattet um die Information aus dem 
    Server auf den Bildschirm anzuzeigen.

    \subsection{Eingabemöglichkeit}
    Um mit dem Pordukt zu interagieren wurde versucht das Resistive Touch-Panel,
    welches bereits auf dem Display vormontiert war, zu verwenden. Durch Schwerigkeiten 
    bei der Implementierung in der Software, wurde sich jetzt für einen Rotary-Encoder 
    als Input entschieden. Es wird dann durch Drehen und Drücken durch das Menü navigiert.

    \subsection{Display}
    Das Display ist mit dem SDD1963 Controller bestückt und kann über 
    einen 16-Bit oder 8-Bit Bus angesteuert werden, letzteres wurde in der 
    im Projekt implementiert. Das Display wurde für seine Größe(7 Zoll) und wegen der Farbdarstellungen ausgewählt.

    \subsection{ARGB}
    Für eine intuitive Benutzung wird zusätzlich ein Piezo fur Aukustische Signale 
    sowie ARGB Leds für Optische Signale eingeplant. Diese Sollen den Anwender durch 
    Licht/Audio einen groben Überblick geben, auch wenn das Control-Panel weiter entfernt ist.
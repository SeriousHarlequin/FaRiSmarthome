\section{Hardware}


    \subsection{MCU}
    Es wurde entschieden einen ESP32-S3 als Microcontroller zu verwenden.
    Dieser ist mit diversen Funktionien(Bluetooth, WLAN, ESPNOW ...) 
    sowie genügend Port-Pins ausgestattet um den Bildschirm parallel mit 8-Bit anzusteuern.

    \subsection{Eingabemöglichkeit}
    Um mit dem Produkt zu interagieren wurde versucht, das Resistive Touch-Panel,
    welches bereits auf dem Display vormontiert war, zu verwenden. Durch Schwierigkeiten 
    bei der Implementierung in der Software, kam es dazu, dass stattdessen ein Drehgeber verwendet
    wurde. Auf diesen kann dann durch Drehen und Drücken durch das Menü navigiert werden.

    \subsection{Display}
    Das Display ist mit dem SSD1963 Controller bestückt und kann über 
    einen 16-Bit oder 8-Bit Bus angesteuert werden. Letzteres wurde
    im Projekt implementiert. Das Display wurde wegen dessen Größe(7 Zoll) und weil es Farben
    darstellen kann ausgewählt.

    \subsection{ARGB}
    Für eine intuitive Benutzung wird zusätzlich ein Piezo-Buzzer für akustische Signale 
    sowie ARGB Leds für optische Signale eingeplant. Diese sollen durch 
    Licht/Audio einen groben Überblick geben, auch wenn das Control-Panel weiter entfernt ist.

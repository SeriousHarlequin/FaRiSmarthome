\section{Software}
Zur Implementierung der Funktionalitäten des Control-Panels werden
verschiedene Software-Komponenten benötigt. Diese sind in den 
folgenden Abschnitten beschrieben.

    \subsection{ESP-NOW}
    Hierbei handelt es sich um ein Protokoll, welches
    speziell für die Kommunikation zwischen ESP-Chips
    entwickelt wurde. Da die Chips selber als Access-Points
    fungieren, ist kein bestehendes Netzwerk notwendig. Unter 
    idealen Bedingungen ist eine Reichweite von bis zu 200m 
    möglich, allerdings nimmt diese in geschlossenen Räumen
    deutlich ab. Im Notfall muss also ein ESP32 als Repeater
    eingesetzt werden. \cite{ESP-NOW-Reliability}
    
    \subsection{Display} 
    Zum Anzeigen der Benutzeroberfläche, sowie um Eingaben zu 
    registrieren, wird der \textit{SquarLineStudio}-UI Editor verwendet.
    Dieser ermöglicht es, die Oberfläche des Control-Panels zu
    designen und anschließend in C-Code zu exportieren.
    Der Code basiert sowohl auf der "TFT\_eSPI" \space Library, welche die
    Ansteuerung des Displays übernimmt, als auch der "lvgl" \space Library,
    welche die Darstellung komplexerer UI-Elemente ermöglicht.

    \subsubsection{Input}
    Die Möglichkeit, Touch-Events zu registrieren und zu verarbeiten ist 
    in der "tft\_eSPI" \space Library integriert. 
    Da das Registrieren des Touches bei seriellen Schnitstellen nicht unterstützt 
    ist, wird stattdessen die "XPT2046\_Touchscreen" Library verwendet. 
    Der Output dieser soll anschließend an die "TFT\_eSPI" Library weitergeleitet
    werden.
    \\~\\
    Da die Touch-Eingabe letztlich nicht implementiert wurde 
    (siehe Kapitel \ref{sec:touch}), ist stattdessen ein Rotary-Encoder
    zur Navigation durch die Oberfläche vorgesehen. Dieser wurde manuell 
    mit dem generierten Code integriert.

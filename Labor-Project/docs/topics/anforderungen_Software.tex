        \section{Software}
        Zur effizienten Realisierung des Panels sind zwei 
        Hauptfunktionalitäten zu implementieren.

            \subsection{MQTT}
            MQTT ist ein Protokoll, welches über TCP/IP arbeitet
            und besonders für IoT Anwendungen geeignet ist. Der 
            Raspberry Pi fungiert als Broker, welcher die
            Kommunikation zwischen den verschiedenen Geräten
            ermöglicht.

            

            \subsection{ESP-NOW}
            Eine mögliche Alternative zu MQTT ist ESP-NOW.
            Hierbei handelt es sich um ein Protokoll, welches
            speziell für die Kommunikation zwischen ESP-Chips
            entwickelt wurde. Da die Chips selber als Access-Points
            fungieren, ist kein bestehendes Netzwerk notwendig. Unter 
            idealen Bedingungen ist eine Reichweite von bis zu 200m 
            möglich.
            
            \subsection{Display}
            Bei der Umsetzung des Displays sind zwei Funktionalitäten
            die es zu implementieren gilt. Zum einen die Anzeige von
            Informationen und zum anderen die Interaktion mit dem
            User Interface. 
            \subsection{UI-Library}
            Für die Ansteuerung des Displays wurde die "TFT\_eSPI"
            \space Library von "Bodmer" verwendet. Diese Library bietet
            Support für den am Display verwendeten Controller.
            
                \subsubsection{Input}
                Die Library bietet die Möglichkeit, Touch-Events zu
                zu registrieren und zu verarbeiten. Da das registrieren
                des Touches nur über SPI möglich ist, wird die
                "XPT2046\_Touchscreen" Library verwendet. Der Output
                soll anschließend an die "TFT\_eSPI" Library weitergeleitet
                werden.

            \subsection{Display}
            Bei der Umsetzung des Displays sind zwei Funktionalitäten
            die es zu implementieren gilt. Zum einen die Anzeige von
            Informationen und zum anderen die Interaktion mit dem
            User Interface. 
        
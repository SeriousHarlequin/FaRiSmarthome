\graphicspath{ {HW-PIC/} }


\chapter{Hardware}

    \section{Umsetzung - V1}
    
    \subsection{V1}
        \subsubsection{Versorgung}
        Das Board wird mit 5V über USB-C versorgt, um den ESP32-S3 mit 3V3 zu 
        versorgung, wird ein AMS1117 Spannungswandler verwenden. Dieser wurde ausgewählt, 
        aufgrund der einfachen Implementierung. Die Verluste bei der Umwandlung können vernachlässigt werden, da 
        es sich nicht um ein batteriebetribenes Gerät handelt.
            \begin{figure}[h]
                \centering
                \includegraphics[width=8cm]{powersupply}
                \caption{AMS1117 mit 3V3 Ausgangsspannung.}
                \label{fig:sch1}
                
            \end{figure}

        Ein- und Ausgangsseitig liegt jeweils ein Glättungskondensator. Die Diode wurde hinzugefügt 
        um zu verhindern, dass am Ausgang eine größere Spannung
        anliegt als beim Eingang - wie z.B. beim Abstecken von dem Gerät.

        \newpage
        \subsubsection{Display}
        Das Display wird parallel im 8-BIT Modus angesteuert, die Pins mit denen 
        kommuniziert wird sind nicht fix in der Libary festgelegt und können am Esp32-S3 
        frei ausgewählt werden. 

            \begin{figure}[h!]
                \centering
                \includegraphics[width=8cm]{connector.png}
                \caption{Stecker für Display und Touch}
                \label{fig:sch2}

            \end{figure}

        Das zugekaufte Display hat ebenfalls ein SD-Kartenleser eingebaut, dieser wird 
        über SPI ausgelesen(Gelb markiert), ist aber für dieses Projekt nicht in der verwendung. 


        \subsubsection{Touch}
        Die Berührung am Resisivtouch-Panel wird vom XPT2046 erfasst und mittels 
        SPI-BUS vom µC ausgelesen. Da es bei der Implementierung von Touch softwareseitig
        Probleme gab, wurde sich für eine Eingabe mit einem Rotary Encoder entschieden.
        Dafür wurden die in der Abbildung \ref{fig:sch2} gezeigten Pins mit dem Prefix "Touch" 
        für das Einlesen des Drehgebers verwendet.

        \subsubsection{Rotary Encoder}

\section{Web Server} \label{sec:webServer}
This section discusses the implementation of the web server
on the Raspberry Pi. The web server is responsible for
serving the web application to the user and for handling
HTTP requests from the user.

        \subsection{Frameworks}
        To create a web server, a suitable framework needs
        to be chosen. The following frameworks were considered:

            \subsubsection{Flask}
            Flask is a lightweight WSGI web application framework.
            It's designed to make getting started quick and easy. 
            It's one of the most popular Python web frameworks and
            can virtually use any Python library including some
            to control GPIO pins on the Raspberry Pi.

            Compared to modern JavaScript frameworks such as Nuxt.js, 
            Flask is considered to be less powerful and less 
            feature-rich.

            \subsubsection{Nuxt.js}
            Nuxt.js is a higher-level framework built on top of Vue.js.
            It's designed to make web development simple and powerful.
            It abstracts away a lot of complexity when it comes to 
            routing and UI rendering.
            
        \subsection{Installing Nuxt.js}
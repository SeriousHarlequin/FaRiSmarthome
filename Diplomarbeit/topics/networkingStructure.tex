\chapter{Networking}
The following chapter will outline the networking structure of 
the project. It describes how the different components of the
smart home will communicate with each other and the user.

\section{Overview}

\begin{figure}[H]
    \centering
    \includegraphics[width=0.8\textwidth]{ThinkCorrectly.jpg}
    \caption{Networking Structure}
\end{figure}
    \subsection{Web Server}
    A Nuxt.js Web server will function as the main interface
    for the user. It will communicate with the database and the
    dedicated Bridge-Node via HTTP. It should be capable of displaying
    and changing the state of the devices in the smart home.
    \subsection{Database}
    The database will store device states and measurements.
    Is can be read and written to by both the Web server and the
    Bridge-Node. 
    \subsection{Bridge-Node}
    The Bridge-Node will act as a bridge between everything running
    on the Raspberry Pi (Web server, database) and the ESP-NOW devices.
    It will be capable of receiving and sending data via both HTTP and
    ESP-NOW. It will also be responsible for discovering new ESP-NOW
    devices.

\section{Communication Protocols}
    \subsection{MQTT}
    MQTT is an OASIS standard messaging protocol for 
    IoT-Applications that builds on top of TCP/IP and 
    works on a publish/subscribe model, which makes it easy
    to set up communication between devices without knowing
    their IP-Addresses. It requires a dedicated broker to
    function, that can be set up as a service on the 
    Raspberry Pi \cite{mqtt_nodate}.
    One mayor upside of using MQTT is that it is compatible
    with every, in this smart home involved, device.


    \subsection{ESP NOW} 
    ESP-NOW is a wireless communication protocol capable of
    working with Wi-Fi and Bluetooth. It supports a 
    multitude of Espressif microcontrollers and is widely 
    used in IoT-Applications. The main advantage of using
    ESP-NOW is that is requires no router or existing
    network to function meaning it's not dependent on
    a good Wi-Fi coverage. While indoor use drastically
    reduces the range of 200m per unit depending on various
    factors \cite{esp-now-reach_2024} 
    the integrated multihop feature that allows the devices 
    to relay messages to each other can be used to extend 
    the range to hard-to-reach locations. One mayor downside
    of the protocol is that one peer can only communicate with
    up to 20 other peers.
    \\~\\
    Due to the listed advantages, ESP-NOW is the \textbf{preferred
    communication protocol} for this project.


\section{Node-to-Server Communication}
Due to the fact that ESP-NOW is exclusive to Espressif microcontrollers,
a bridge between the nodes and the server is needed. This bridge can be 
routed through a preexisting Network as the Web server requires 
one anyway.


    \subsection{MQTT}
    MQTT can be used to bridge the gap between the nodes and the server.
    The server can subscribe to the same topics as the nodes, and the nodes
    can publish to the same topics as the server.

        \subsubsection{Disadvantages}
        Unfortunately, MQTT integration for the Web server as well as the database
        is not as straightforward as the alternative and would require a lot more 
        overhead.

    \subsection{HTTP}
    HTTP can be used to bridge the gap between the nodes and the server.
    This way a dedicated Node can be set up to handle the communication
    with both the Web server and the database and distribute it to the 
    rest of the nodes.

    \vspace{1cm}
    \textbf{For this project, HTTP is the preferred protocol for 
    node-to-server communication.}

    \subsection{Network Discovery}
    For discovering the Raspberry Pi on the ESP and vice versa,
    IP's need to be exchanged. This was tried in two ways:
        \subsubsection{By Hostname}
        The first approach was to set a hostname for the 
        ESP32 and fetch data through it. While this worked
        in most networks, it isn't possible or needs to 
        configured in some, which would make the average 
        consumer unable to make use of the system.

        \subsubsection{Via Broadcast}
        The second approach was to broadcast the IP of the ESP32
        at the press of a button. This would then send an 
        HTTP-GET-Request into the network and the Raspberry Pi
        would respond with its IP. While sending a broadcast is
        a lot of network traffic, it is the most reliable way
        to discover the Raspberry Pi in any network. To minimize
        unintended traffic, the broadcast is only sent once when the
        button is pressed.
        

\section{Node-to-Node Communication}
    \subsection{ESP NOW Discovery}
    To discover other ESP-NOW devices, a broadcast can be sent
    from the Bridge-Node to all other ESP-NOW devices. This
    Way every ESP-NOW device can discover the bridge and send
    their MAC-Address in return.

    \subsection{Circumventing the Peer Limit}
    To circumvent the 20 peer limit, the bridge can dynamically
    dedicate another peer to act as a second bridge should the
    need arise. 




    
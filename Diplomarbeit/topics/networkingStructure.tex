\chapter{Networking}
\section{Networking Structure}


\section{Node-to-Node Communication}
    \subsection{MQTT}
    MQTT is an open source broker-based protocol built on top of TCP/IP. It
    requires a dedicated broker to function. Once set up a client can simply 
    publish or subscribe to a topic, and receive/send messages to/from other 
    clients.
    
        \subsubsection{Advantages}
        \begin{itemize}
            \item Compatible with most microcontrollers as well as the Web server.
            \item IP or MAC addresses do not need to be exchanged for communication.
        \end{itemize}

        \subsubsection{Disadvantages}
        \begin{itemize}
            \item Requires a dedicated broker.
            \item Requires a preexisting network.
            \item Note position is limited by the range of the network.
        \end{itemize}

    \subsection{ESP-NOW}
    ESP-NOW is a protocol developed by Espressif Systems for the ESP32 and ESP8266
    microcontrollers. It is a peer-to-peer protocol, meaning that it does not require
    a dedicated server to function.

        \subsubsection{Advantages}
        \begin{itemize}
            \item No need for a dedicated broker.
            \item No need for a preexisting network.
            \item Range can be extended by creating a mesh network.
        \end{itemize}

        \subsubsection{Disadvantages}
        \begin{itemize}
            \item Limited to ESP32 and ESP8266 microcontrollers.
            \item Limited to a range of 200 meters.
        \end{itemize}

    \vspace{1cm}
    \textbf{For this project, ESP-NOW is the preferred protocol for 
    node-to-node communication.}

\section{Node-to-Server Communication}
Due to the fact that ESP-NOW is exclusive to Espressif microcontrollers,
a bridge between the nodes and the server is needed. This bridge can be 
routed through a preexisting Network as the Web server requires 
one anyway.

    \subsection{MQTT}
    MQTT can be used to bridge the gap between the nodes and the server.
    The server can subscribe to the same topics as the nodes, and the nodes
    can publish to the same topics as the server.

        \subsubsection{Disadvantages}
        Unfortunately, MQTT integration for the Web server as well as the database
        is not as straightforward as the alternative and would require a lot more 
        overhead.

    \subsection{HTTP}
    HTTP can be used to bridge the gap between the nodes and the server.
    This way a dedicated Node can be set up to handle the communication
    with both the Web server and the database and distribute it to the 
    rest of the nodes.

    \vspace{1cm}
    \textbf{For this project, HTTP is the preferred protocol for 
    node-to-server communication.}




\chapter{Networking}
\section{Communication Protocols}
    \subsection{MQTT}
    MQTT is an OASIS standard messaging protocol for 
    IoT-Applications that builds on top of TCP/IP and 
    works on a publish/subscribe model, which makes it easy
    to set up communication between devices without knowing
    their IP-Addresses. It requires a dedicated broker to
    function, that can be set up as a service on the 
    Raspberry Pi \cite{mqtt_nodate}.
    One mayor upside of using MQTT is that it is compatible
    with every, in this smart home involved, device.


    \subsection{ESP NOW} 
    ESP-NOW is a wireless communication protocol capable of
    working with Wi-Fi and Bluetooth. It supports a 
    multitude of Espressif microcontrollers and is widely 
    used in IoT-Applications. The main advantage of using
    ESP-NOW is that is requires no router or existing
    network to function meaning it's not dependent on
    a good Wi-Fi coverage. While indoor use drastically
    reduces the range of 200m per unit depending on various
    factors \cite{esp-now-reach_2024} 
    the integrated multihop feature that allows the devices 
    to relay messages to each other can be used to extend 
    the range to hard-to-reach locations.\\~\\
    Due to the listed advantages, ESP-NOW is the \textbf{preferred
    communication protocol} for this project.

\section{Networking Structure}
The Nuxt.js Webserver will function as a main broker between
the Database and a dedicated ESP32 device that  acts
as a bridge between every ESP32 and the Nuxt.js Webserver.
This is done because a single ESP32 device can only handle
either ESP-NOW or a standard Wi-Fi connection, but not both
simultaneously. This dedicated bridge is a special 
ESP32-Board with both a Wi-Fi antenna and an Ethernet port,
enabling it to do both at the same time. 


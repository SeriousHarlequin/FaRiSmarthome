\section{Replicating the operating system}
This section discusses the process of replicating the
operating system of the Raspberry Pi. There are three options
to consider. 
    \subsection{Image}
    Making an image of the operating system is the most
    straightforward method. The image can be created using
    the following command:
    \begin{minted}{bash}
sudo dd if=/dev/sdX of=~/image.img bs=4M
    \end{minted}
    While this method is simple, it has some drawbacks. The
    image will be the same size as the SD card, even if the
    card is mostly empty. This can lead to large image files
    that take up unnecessary space. Additionally, you are
    locked into the same Raspberry Pi model and partition
    size.

    \subsection{Setup Script}
    A setup script can be used to replicate the operating
    system. This script can be written in any language, but
    bash is the most common. The script will install all the
    necessary packages and configurations. This method is
    more flexible than creating an image, but it requires
    more work.

    \subsection{Containerization}
    The best solution is to use containerization. Docker is
    the most popular containerization tool, it allows
    you to create a container that contains all the necessary
    packages and configurations. This container can then be
    run on any Computer that has Docker installed. 

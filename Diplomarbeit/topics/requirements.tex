\chapter{Requirements} \setAuthor{Fabian Schätzschock}
The following chapter will outline the requirements for the 
project. It describes what concepts and technologies are needed
to implement the project.
    \section{Hardware}
    \section{Software}
        \subsection{Microcontroller}
            \subsubsection{IDE}
            For developing purposes an IDE fulfilling the following criteria is needed:
            \begin{itemize}
                \item \textbf{Integrates the Arduino framework:}
                Needs to integrate Arduino as most of the code will be written using it.
                \item \textbf{Support for different microcontrollers:}
                Supporting different microcontrollers, to allow for flexibility
                in developing new Nodes.
                \item \textbf{Manages Library integration:}
                Streamlines installing and managing libraries used in the project.
                \item \textbf{Free and Open Source:}
                To allow anyone to easily modify  or create their own smart home
                using this project as a base.
                \item \textbf{Compiling and uploading code:} 
                Simplifying the process of uploading code to the microcontroller.
            \end{itemize} 

        \subsection{Communication between Nodes}
        
            \subsubsection{Note-to-Note}
            Communication between nodes should be possible between any two nodes
            both ways.

            \subsubsection{Note-to-Server}
            Communication between the nodes and the web server, as well as the 
            database, should be possible both ways.

        \subsection{Backend}
        To record and read previous data, a database is needed, it should ideally be 
        working with time series data and be able to handle a large amount of data, 
        as it will mainly be used for storing sensor data.

        \subsection{Frontend}
        The frontend should be able to visualize the data stored in the database,
        control the actuators and sensors, and display the current state of the 
        system. To implement this, a high-level web framework is preferred to 
        simplify the development process. 
\chapter{Used technologies} \setAuthor{Fabian Schätzschock}
The following chapter will outline the requirements for the 
project. It describes what concepts and technologies are needed
to implement the project.
    \section{Hardware}
        \subsection{Microcontroller}
        
    \section{Arduino}\label{sec:arduino}

    \section{Development Environment}
    For developing purposes an Enviroment fulfilling the following criteria is needed:
    \begin{itemize}[itemsep=0em]
        \item \textbf{Integrates the Arduino framework:}
        Needs to integrate Arduino as most of the code will be written using it. 
        (see section \ref{sec:arduino})
        \item \textbf{Support for different microcontrollers:}
        Supporting different microcontrollers, to allow for flexibility
        in developing new Nodes.
        \item \textbf{Manages Library integration:}
        Streamlines installing and managing libraries used in the project.
        \item \textbf{Free:}
        To allow anyone to easily modify  or create their own smart home
        using this project as a base.
        \item \textbf{Compiling and uploading code:} 
        Simplifying the process of uploading code to the microcontroller.
        \item \textbf{Edit Code through SSH:}
        Streamline the process of editing and executing code remotely
    \end{itemize} 

        \subsection{Visual Studio Code}
        \subsection{PlatformIO}

    \section{FreeRTOS}
    %mention that Interrupts and Webserver might be a problem on the bridge

    \section{Communication}
    
        \subsection{Note-to-Note}
        Communication between nodes should be possible between the bridge and
        any node, all traffic should be routed through the bridge.
        The following protocols are considered:
            \subsubsection{MQTT}
            MQTT is an OASIS standard messaging protocol for 
            IoT-Applications that builds on top of TCP/IP and 
            works on a publish/subscribe model, which makes it easy
            to set up communication between devices without knowing
            their IP-Addresses. It requires a dedicated broker to
            function, that can be set up as a service on the 
            Raspberry Pi \cite{mqtt_nodate}.
            One mayor upside of using MQTT is that it is compatible
            with every, in this smart home involved, device.
        
        
            \subsubsection{ESP NOW} 
            ESP-NOW is a wireless communication protocol capable of
            working with Wi-Fi and Bluetooth. It supports a 
            multitude of Espressif microcontrollers and is widely 
            used in IoT-Applications. The main advantage of using
            ESP-NOW is that is requires no router or existing
            network to function meaning it's not dependent on
            a good Wi-Fi coverage. While indoor use drastically
            reduces the range of 200m per unit depending on various
            factors \cite{esp-now-reach_2024} 
            the integrated multihop feature that allows the devices 
            to relay messages to each other can be used to extend 
            the range to hard-to-reach locations. One mayor downside
            of the protocol is that one peer can only communicate with
            up to 20 other peers.
            \\~\\
            Due to the listed advantages, ESP-NOW is the \textbf{preferred
            communication protocol} for this project.

        \subsection{Note-to-Server}
        Communication between the nodes and the web server, as well as the 
        database, should be possible both ways.

            \subsubsection{MQTT}
            MQTT can be used to bridge the gap between the nodes and the server.
            The server can subscribe to the same topics as the nodes, and the nodes
            can publish to the same topics as the server.
        
                \textbf{Disadvantages:}
                Unfortunately, MQTT integration for the Web server as well as the database
                is not as straightforward as the alternative and would require a lot more 
                overhead.
        
            \subsubsection{HTTP}
            HTTP can be used to bridge the gap between the nodes and the server.
            This way a dedicated Node can be set up to handle the communication
            with both the Web server and the database and distribute it to the 
            rest of the nodes.
        
            \vspace{1cm}
            \textbf{For this project, HTTP is the preferred protocol for 
            node-to-server communication.}

    \section{Backend}
    To record and read previous data, a database is needed, it should ideally be 
    working with time series data and be able to handle a large amount of data, 
    as it will mainly be used for storing sensor data.

    \section{Frontend}
    The frontend should be able to visualize the data stored in the database,
    control the actuators and sensors, and display the current state of the 
    system. To implement this, a high-level web framework is preferred to 
    simplify the development process. 
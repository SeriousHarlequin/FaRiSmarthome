\chapter{Used technologies} \setAuthor{Fabian Schätzschock}
The following chapter will outline the requirements for the 
project. It describes what concepts and technologies are needed
to implement the project.
    \section{Hardware}
        \subsection{Microcontroller}
        
    \section{Arduino}\label{sec:arduino}
    \begin{wrapfigure}[2]{tr}{0.13\textwidth} % 'r' for right, and width of the image
        \vspace{-1.4cm}
        \hspace{3cm}
        \includegraphics[width=\linewidth]{Arduino_Logo.png}
    \end{wrapfigure}
    The Arduino framework is an open source platform that
    provides a subset of C++ tailored for embedded systems
    with a vast ecosystem of libraries. It is widely used
    and beginner-friendly, making it the perfect choice for
    this project, due to the fact that the smart home 
    should be easy to extend and modify.

    \section{Development Environment}
    For developing purposes an Enviroment fulfilling the following criteria is needed:
    \begin{itemize}[itemsep=0em]
        \item \textbf{Integrates the Arduino framework:}
        Needs to integrate Arduino as most of the code will be written using it. 
        (see section \ref{sec:arduino})
        \item \textbf{Support for different microcontrollers:}
        Supporting different microcontrollers, to allow for flexibility
        in developing new Nodes.
        \item \textbf{Manages Library integration:}
        Streamlines installing and managing libraries used in the project.
        \item \textbf{Free:}
        To allow anyone to easily modify  or create their own smart home
        using this project as a base.
        \item \textbf{Compiling and uploading code:} 
        Simplifying the process of uploading code to the microcontroller.
        \item \textbf{Edit Code through SSH:}
        Streamline the process of editing and executing code remotely
    \end{itemize} 
        \vspace{1cm}
        \subsection{Visual Studio Code}
        \begin{wrapfigure}[3]{tr}{0.13\textwidth} % 'r' for right, and width of the image
            \vspace{-1.4cm}
            \hspace{3cm}
            \includegraphics[width=\linewidth]{VSCode.png}
        \end{wrapfigure}
        Visual Studio Code \cite{visualStudioCode} is a text editor developed by Microsoft that
        offers a wide range of features, including a VSCode server that
        allows for remote development. It also offers a wide range of 
        extensions, including PlatformIO, which transforms it into a 
        powerful IDE for developing microcontroller applications.

        \subsection{PlatformIO}
        \begin{wrapfigure}[3]{tr}{0.13\textwidth} % 'r' for right, and width of the image
            \vspace{-1.4cm}
            \hspace{3cm}
            \includegraphics[width=\linewidth]{PlatformIO.png}
        \end{wrapfigure}
        PlatformIO \cite{platformio} is an open-source ecosystem for microcontroller
        development that supports a wide range of microcontrollers including
        the ESP32 and ESP8266. It supports multiple frameworks like
        Arduino or ESP-IDF and can be integrated into Visual Studio Code,
        CLion or used through the command line. It also eases library inclusion
        and streamlines the compiling and uploading process. 

    \section{FreeRTOS}
        \begin{wrapfigure}[2]{tr}{0.13\textwidth} % 'r' for right, and width of the image
            \vspace{-1.4cm}
            \hspace{3cm}
            \includegraphics[width=\linewidth]{freeRTOS.png}
        \end{wrapfigure}
    FreeRTOS \cite{freertos} is a market-leading open source embedded system RTOS supporting 
    over 40 processor architectures including the ESP32 with a small memory 
    footprint, fast execution times, and cutting-edge RTOS features.
    \par
    A RTOS is necessary for this project as there are a lot of asynchronous
    tasks that need to be handled, like reading sensor data, controlling
    actuators, and handling communication between the nodes. Using interrupts
    would be to complex and error-prone, so the integrated RTOS is the
    preferred solution.

    %mention that Interrupts and Webserver might be a problem on the bridge

    \section{Communication}
    The following section will outline the communication protocols 
    considered and used for the various communication channels.
    
        \subsection{Note-to-Note}
        Communication between nodes should be possible between the bridge and
        any node, all traffic should be routed through the bridge.
        The following protocols are considered:
            \subsubsection{MQTT}
            \begin{wrapfigure}[2]{tr}{0.18\textwidth} % 'r' for right, and width of the image
                \vspace{-1cm}
                \hspace{3cm}
                \includegraphics[width=\linewidth]{mqtt-logo.png}
            \end{wrapfigure}
            MQTT \cite{mqtt_nodate} is an OASIS standard messaging protocol for 
            IoT-Applications that builds on top of TCP/IP and 
            works on a publish/subscribe model, which makes it easy
            to set up communication between devices without knowing
            their IP-Addresses. It requires a dedicated broker to
            function, that can be set up as a service on the 
            Raspberry Pi.
            One mayor upside of using MQTT is that it is compatible
            with every, in this smart home involved, device.
        
        
            \subsubsection{ESP NOW} 
            \begin{wrapfigure}[2]{tr}{0.18\textwidth} % 'r' for right, and width of the image
                \vspace{-1cm}
                \hspace{3cm}
                \includegraphics[width=\linewidth]{espnow-logo.png}
            \end{wrapfigure}
            ESP-NOW is a wireless communication protocol capable of
            working with Wi-Fi and Bluetooth. It supports a 
            multitude of Espressif microcontrollers and is widely 
            used in IoT-Applications. The main advantage of using
            ESP-NOW is that is requires no router or existing
            network to function meaning it's not dependent on
            a good Wi-Fi coverage. While indoor use drastically
            reduces the range of 200m per unit depending on various
            factors \cite{esp-now-reach_2024} 
            the integrated multihop feature that allows the devices 
            to relay messages to each other can be used to extend 
            the range to hard-to-reach locations. One mayor downside
            of the protocol is that one peer can only communicate with
            up to 20 other peers.
            \\~\\
            Due to the listed advantages, ESP-NOW is the \textbf{preferred
            communication protocol} for this project.

        \subsection{Note-to-Server}
        Communication between the nodes and the web server, as well as the 
        database, should be possible both ways.

            \subsubsection{MQTT}
            MQTT can be used to bridge the gap between the nodes and the server.
            The server can subscribe to the same topics as the nodes, and the nodes
            can publish to the same topics as the server.
        
                \textbf{Disadvantages:}
                Unfortunately, MQTT integration for the Web server as well as the database
                is not as straightforward as the alternative and would require a lot more 
                overhead.
        
            \subsubsection{HTTP}
            HTTP can be used to bridge the gap between the nodes and the server.
            This way a dedicated Node can be set up to handle the communication
            with both the Web server and the database and distribute it to the 
            rest of the nodes.
        
            \vspace{1cm}
            \textbf{For this project, HTTP is the preferred protocol for 
            node-to-server communication.}


    \section{Nuxt.js}
    The frontend should be able to visualize the data stored in the database,
    control the actuators and sensors, and display the current state of the 
    system. To implement this, a high-level web framework is preferred to 
    simplify the development process. \npar

    \begin{wrapfigure}[2]{tr}{0.18\textwidth} % 'r' for right, and width of the image
        \vspace{-1cm}
        \hspace{3cm}
        \includegraphics[width=\linewidth]{nuxt-js-logo.jpeg}
    \end{wrapfigure}
    For this project, Nuxt.js \cite{nuxtjs} is used as the web framework.
    Nuxt.js is a framework for creating universal Vue.js applications.
    It is a high-level framework that simplifies the development of
    Vue.js applications by providing a set of conventions and best practices.
    It is built on top of Vue.js and provides a set of features that
    make it easy to create a web application. 

    \section{Flask}
    \begin{wrapfigure}[5]{tr}{0.18\textwidth} % 'r' for right, and width of the image
        \vspace{-1cm}
        \hspace{3cm}
        \includegraphics[width=\linewidth]{flask-logo.png}
    \end{wrapfigure}
    Flask is a lightweight WSGI web application framework. It is 
    designed to make getting started quick and easy. \cite{flask}
    In this thesis, Flask is not used in the traditional way,
    as it is not used to serve the web application, but rather
    to serve as a receiver of UDP packets from the bridge node
    (see section \ref{sec:network_discovery}). This is done 
    because Python offers a lot of libraries for listening to
    UDP packets and parsing the data. This data is then accessed
    through an HTTP-GET request from Nuxt.js when needed.

    
    \section{Docker}
    \begin{wrapfigure}[4]{tr}{0.18\textwidth} % 'r' for right, and width of the image
        \vspace{-1cm}
        \hspace{3cm}
        \includegraphics[width=\linewidth]{docker-logo.png}
    \end{wrapfigure}
    Docker is an open platform for developing, shipping, and running 
    applications. Docker enables you to separate your applications 
    from your infrastructure so you can deliver software quickly. 
    With Docker, you can manage your infrastructure in the same ways 
    you manage your applications. By taking advantage of Docker's 
    methodologies for shipping, testing, and deploying code, you can 
    significantly reduce the delay between writing code and running 
    it in production. \cite{what_is_docker}


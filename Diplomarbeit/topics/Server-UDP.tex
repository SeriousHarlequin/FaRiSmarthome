\section{UDP Process}
The Raspberry Pi runs a python process starting at boot with
the purpose of listening to a UDP broadcast sent by the bridge
node. It then communicates the IP address of the bride to the
web server when requested.

    \subsection{Start on boot}
    The UDP process is started on boot by a systemd service.
    This section describes how to add a systemd service to the
    Raspberry Pi.

    \begin{enumerate}
        \item To add an systemd entry first create a systemd service file: \\
        \texttt{\$ sudo vi /etc/systemd/system/upd\_discovery.service}

        \item Then add the following in the file:
        \begin{lstlisting}[style=cppCode]
[Unit]
Description=udp discovery
After=network.target

[Service]
ExecStart=/usr/bin/python3 /home/fabian/farismart/Main.py
WorkingDirectory=/home/fabian
StandardOutput=inherit
StandardError=inherit
Restart=always
User=fabian

[Install]
WantedBy=multi-user.target
        \end{lstlisting}

        \item Enable the service: \\
        \texttt{\$ sudo systemctl enable udp\_discovery.service}
    \end{enumerate}
    
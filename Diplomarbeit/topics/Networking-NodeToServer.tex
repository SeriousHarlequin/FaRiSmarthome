\section{Node-to-Server Communication}
This section discusses important aspects or problems of the communication between
the nodes and the server. Note that the implementation of these aspects is discussed in section
\ref{sec:farilib_espnow}.

\vspace{0.5cm}

Due to the fact that ESP-NOW is exclusive to Espressif microcontrollers,
a bridge between the nodes and the server is needed. This bridge can be 
routed through a preexisting Network as the Web server requires 
one anyway.


    \subsection{MQTT}
    MQTT can be used to bridge the gap between the nodes and the server.
    The server can subscribe to the same topics as the nodes, and the nodes
    can publish to the same topics as the server.

        \subsubsection{Disadvantages}
        Unfortunately, MQTT integration for the Web server as well as the database
        is not as straightforward as the alternative and would require a lot more 
        overhead.

    \subsection{HTTP}
    HTTP can be used to bridge the gap between the nodes and the server.
    This way a dedicated Node can be set up to handle the communication
    with both the Web server and the database and distribute it to the 
    rest of the nodes.

    \vspace{1cm}
    \textbf{For this project, HTTP is the preferred protocol for 
    node-to-server communication.}

    \subsection{Network Discovery}
    For discovering the Raspberry Pi on the ESP and vice versa,
    IP's need to be exchanged. This was tried in two ways:
        \subsubsection{By Hostname}
        The first approach was to set a hostname for the 
        ESP32 and fetch data through it. While this worked
        in most networks, it isn't possible or needs to 
        configured in some, which would make the average 
        consumer unable to make use of the system.

        \subsubsection{Via Broadcast}
        The second approach was to broadcast the IP of the ESP32
        at the press of a button. This would then send an 
        HTTP-GET-Request into the network and the Raspberry Pi
        would respond with its IP. While sending a broadcast is
        a lot of network traffic, it is the most reliable way
        to discover the Raspberry Pi in any network. To minimize
        unintended traffic, the broadcast is only sent once when the
        button is pressed.
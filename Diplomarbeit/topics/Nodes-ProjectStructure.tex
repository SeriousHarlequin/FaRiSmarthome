\section{Project Structure} \label{sec:node_project_structure}

This section describes the project structure of the nodes
and how functionality is meant to be extended. Every Project
should make use of the FaRiLib library (see section \ref{sec:farilib})
and follow a similar structure.

    \subsection{Slave Node}
    Ideally there should only be one slave binary that is able to 
    run on all slave Nodes. This means, unless there is a good 
    reason, new functionality should be implemented into the preexisting
    slave project.
        \subsubsection{Adding new Functionality}
        To add new functionality to the slave node, the following steps
        should be taken:
        \begin{enumerate}
            \item Create a new task that handles the new functionality
            \item Add the new case to the switch statement in the
            \texttt{Setup} function choosing what tasks to activate
            \item Adjust the Web server to display the data
        \end{enumerate}

    \subsection{Bridge Node}
    While it is possible to have more devices communicating with
    the web server, the recommended course of action is to have
    one bridge node that handles all communication.
    This means only one binary is needed for the bridge node.
        \subsubsection{Adding new Functionality}
        For adding new Nodes communicating via ESP-NOW to the smart 
        home system, there is only minimal restructuring of the
        source code necessary. \\~\\
        If the new Node is not a ESP32, a new communication protocol
        needs to be implemented. While this is not a trivial task,
        it is possible to extend the FaRiLib library without complete
        refactoring. \\~\\
        Basic functionality can be added in the following files:
        \begin{itemize}
            \item \texttt{main.cpp}: Add a task or process an 
            incoming message
            \item \texttt{WebServer.cpp}: Add a new HTTP endpoint
        \end{itemize}
        
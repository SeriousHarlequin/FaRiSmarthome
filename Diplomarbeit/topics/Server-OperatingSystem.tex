\section{Operating System}
In order to find a suitable operating system for the Raspberry Pi, a few options
were considered. The requirements for the operating system are as follows:
\begin{itemize}
    \item \textbf{Lightweight:} Should require as little resources as possible
    and not come with unnecessary software preinstalled. This includes a Desktop 
    Environment.
    \item \textbf{Software Support:} Has to support the software needed for the
    project, such as a \texttt{Nuxt.js} web server (section \ref{sec:webServer})
    and an \texttt{InfluxDB} database (section \ref{sec:database}).
    \item \textbf{ARM Based:} As the Raspberry Pi uses an ARM processor, the
    operating system needs to be compatible with it.
    \item \textbf{Reproducibility:} Should offer a way to replicate the operating
    system on other Raspberry Pi's.
\end{itemize}
    \subsection{Raspberry Pi OS Lite}
    Raspberry Pi OS Lite is a lightweight version of the Raspberry Pi OS. It
    comes without many packages including but not limited to a Desktop Environment.
    It supports all the software needed for the project and offers extensive documentation.
    \subsection{Other ARM-based OS}
    Many other Linux distros offer a version for ARM processors. For the purpose of
    this project, there was no need to look further than the Raspberry Pi OS Lite.
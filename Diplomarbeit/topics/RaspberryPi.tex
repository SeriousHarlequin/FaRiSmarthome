\chapter{Raspberry Pi} \setAuthor{Fabian Schätzschock} \label{sec:raspberry_pi}
This chapter discusses the software side of the Raspberry Pi,
including the chosen software and operating system, as well 
as the implementation of the web server and database.
    \section{Operating System}
    In order to find a suitable operating system for the Raspberry Pi, a few options
    were considered. The requirements for the operating system are as follows:
    \begin{itemize}
        \item \textbf{Lightweight:} Should require as little resources as possible
        and not come with unnecessary software preinstalled. This includes a Desktop 
        Environment.
        \item \textbf{Software Support:} Has to support the software needed for the
        project, such as a \texttt{Nuxt.js} web server (section \ref{sec:webServer})
        and an \texttt{InfluxDB} database (section \ref{sec:database}).
        \item \textbf{ARM Based:} As the Raspberry Pi uses an ARM processor, the
        operating system needs to be compatible with it.
        \item \textbf{Reproducibility:} Should offer a way to replicate the operating
        system on other Raspberry Pi's.
    \end{itemize}
        \subsection{Raspberry Pi OS Lite}
        Raspberry Pi OS Lite is a lightweight version of the Raspberry Pi OS. It
        comes without many packages including but not limited to a Desktop Environment.
        It supports all the software needed for the project and offers extensive documentation.
        \subsection{Other ARM-based OS}
        Many other Linux distros offer a version for ARM processors. For the purpose of
        this project, there was no need to look further than the Raspberry Pi OS Lite.
    \section{Raspberry to ESP Communication}
    \section{Database} \label{sec:database}
    \section{Web Server} \label{sec:webServer}
        \subsection{Frameworks}
        To create a web server, a suitable framework needs
        to be chosen. The following frameworks were considered:

            \subsubsection{Flask}
            Flask is a lightweight WSGI web application framework.
            It's designed to make getting started quick and easy. 
            It's one of the most popular Python web frameworks and
            can virtually use any Python library including some
            to control GPIO pins on the Raspberry Pi.

            Compared to modern JavaScript frameworks such as Nuxt.js, 
            Flask is considered to be less powerful and less 
            feature-rich.

            \subsubsection{Nuxt.js}
            Nuxt.js is a higher-level framework built on top of Vue.js.
            It's designed to make web development simple and powerful.
            It abstracts away a lot of complexity when it comes to 
            routing and UI rendering.
            
        \subsection{Installing Nuxt.js}

    \section{Replicating the operating system}
        \subsection{Setup Script}
        \subsection{Containerization}

\chapter{Raspberry Pi}
This chapter discusses the software side of the Raspberry Pi,
including the chosen software and operating system, as well 
as the implementation of the web server and database.
    \section{Operating System}
        \subsection{Raspberry Pi OS Lite}
        \subsection{Other ARM-based OS}
    \section{Raspberry to ESP Communication}
    \section{Database}
    \section{Web Server}
        \subsection{Frameworks}
        To create a web server, a suitable framework needs
        to be chosen. The following frameworks were considered:

            \subsubsection{Flask}
            Flask is a lightweight WSGI web application framework.
            It's designed to make getting started quick and easy. 
            It's one of the most popular Python web frameworks and
            can virtually use any Python library including some
            to control GPIO pins on the Raspberry Pi.

            Compared to modern JavaScript frameworks such as Nuxt.js, 
            Flask is considered to be less powerful and less 
            feature-rich.

            \subsubsection{Nuxt.js}
            Nuxt.js is a higher-level framework built on top of Vue.js.
            It's designed to make web development simple and powerful.
            It abstracts away a lot of complexity when it comes to 
            routing and UI rendering.
            
        \subsection{Installing Nuxt.js}

    \section{Replicating the operating system}
        \subsection{Setup Script}
        \subsection{Containerization}

\subsection{Hardware - V1.1}
    Version 1.X of the basic module was inteeded to be a ruff prototype to test the concept of a 
    plugable battery powered module. For this reason only the battery module was designed for this
    version. Version 1.0 was never manifactured, but the design was used as a base for the
    version 1.1. The version 1.1 was manifactured and tested. 

    \subsubsection{Module-Connector}
        The module connector had three connectors, one for the battery and two for the
        module. The battery connector has a 5 pin header. 

        \begin{figure}[H]
            \centering
            \includegraphics[width=0.6\textwidth]{assets/HW/TBD2.png}
            \caption{Module-Connector implemented in the schematic.}
        \end{figure}


        To connect a module there is a 10 pin header with IO-ports and a 6 pin header 
        for suppling power. It was not inteeded to design a module, besides the battery module,
        for this version. It was planned to be used in testing as an development board and connecting
        the sensors over jumper wires.

    \subsubsection{DC-DC Converter}
        
        To convert the 5V from the USB port to 3.3V a LM7805 linear regulator was used. This regulator
        would then be bybassed by the battery module when connected. Then the 3.3V would be supplied by the
        battery module. The 5V was also used to charge the battery.

        The LM7805 was turned off by a P-Channel MOSFET when the battery module was connected.

    \subsubsection{Seriel-To-USB Converter}
        The Seriel-To-USB converter was deamed to be unnassary since the MCU could also be programmed 
        over native USB, it was removed from later versions.

    \subsubsection{PCB}

\subsection{Hardware - V1.0}